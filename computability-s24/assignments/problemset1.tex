\documentclass[12pt]{article}
% 12pt font size, article class.
% Usually stick to this, though you can try "amsart" for an AMS style article instead.

\usepackage{amsmath,amssymb,psfrag,epsfig,boxedminipage,helvet,amsthm,endnotes,version,multicol}
\usepackage{pgfplots}
\usepackage{tikz}
\usetikzlibrary{automata, positioning, arrows}
% Package imports. We probably don't need all of them, but it doesn't hurt.

\title{Problem Set 1}
\author{Your Name Here!}
\date{}

\makeatletter
\let\newtitle\@title
\makeatother

\renewcommand{\thepage}{\footnotesize MAT 4520, Computability, \newtitle, p. \arabic{page}} 

\newcommand{\sol}{\par{\bf SOLUTION}: }

\begin{document}

\maketitle

\noindent
    {\bf Due: Thursday, 2/17, at 11:59pm. }
        
For all of these problems, the alphabet $\Sigma = \{ 0, 1 \}$.
    
\begin{enumerate}
\item Let $\mathcal{L}_1 = \{ w : w$ starts with a 0 and has odd length $\}$.
    \begin{enumerate}
    \item Give a state diagram of a DFA that accepts $\mathcal{L}_1$.
    \item Give the formal description of the DFA as well.
    \end{enumerate} 

\sol % Put your solution here!
\begin{enumerate}
	\item Solution for (a)
	\item Solution for (b)
\end{enumerate}
    
\item Consider the language $\mathcal{L}_2 = \{ (01)^n : n \in \mathbb{N} \}$.
\begin{enumerate}
	\item Give two examples of words in $\mathcal{L}_2$ and two examples of words not in $\mathcal{L}_2$.
	\item Give a state diagram of a DFA which accepts $\mathcal{L}_2$.
\end{enumerate}

\sol % Put your solution here!
\begin{enumerate}
	\item Solution for (a)
	\item Solution for (b)
\end{enumerate}

\item Show that if $\mathcal{L}_1$ and $\mathcal{L}_2$ are regular languages, then so is $\mathcal{L}_1 \cap \mathcal{L}_2$. That is: prove that the class of regular languages is closed under intersection.

\sol % Put your solution here!


\item \label{l4} Let $\mathcal{L}_3 = \{ xy : x$ and $y$ are words over $\Sigma$, $x$ starts with a $0$, and $y$ starts with a $1 \}.$ Show that $\mathcal{L}_3$ is regular by giving a state diagram of an NFA which recognizes it.

\sol % Put your solution here!


\item Convert your NFA in question (\ref{l4}) to a DFA using the algorithm we described in class.

\sol % Put your solution here!

%%% LaTeX HINTS:

% If you want to "draw" diagrams in LaTeX, you have two options:

%1. Use tikz. This can be hard, but you can look at the example from Pre-Work Lesson 2, copied over here.
%	\begin{figure}[ht]
%	\centering
%	\begin{tikzpicture}
%	\node[state, initial] (q0) {$q_0$};
%	\node[state, right of=q0] (q1) {$q_1$};
%	\node[state, accepting, right of=q1] (q2) {$q_2$};
%	\draw (q0) edge[above] node{0, 1} (q1)
%	(q1) edge[above] node{0, 1} (q2)
%	(q2) edge[bend left, below] node{0, 1} (q0);
%	\end{tikzpicture}
%	\end{figure}

%2. Draw it by hand. Take a good picture, crop it, and upload it to overleaf. Then use the following command:

% \includegraphics{filename-without-extension}
% Depending on the dimensions of the picture, you might need to rescale it:
% \includegraphics[scale=0.5]{filename-without-extension}.
% Sometimes I need to use scale=0.2 or 0.3 before the pictures display decently.

\end{enumerate}

\end{document}