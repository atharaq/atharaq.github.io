\documentclass[12pt]{article}
% 12pt font size, article class.
% Usually stick to this, though you can try "amsart" for an AMS style article instead.

\usepackage{amsmath,amssymb,psfrag,epsfig,boxedminipage,helvet,amsthm,endnotes,version,multicol}
\usepackage{pgfplots}
\usepackage{tikz}
\usetikzlibrary{automata, positioning, arrows}
% Package imports. We probably don't need all of them, but it doesn't hurt.

\title{Problem Set 2}
\author{Your Name Here!}
\date{}

\makeatletter
\let\newtitle\@title
\makeatother

\renewcommand{\thepage}{\footnotesize MAT 4520, Computability, \newtitle, p. \arabic{page}} 

\newcommand{\sol}{\par{\bf SOLUTION}: }

\begin{document}

\maketitle

\noindent
    {\bf Due: Monday, 2/26, at 11:59pm. }
    
\begin{enumerate}
\item Fill in the details for the proof, mentioned in lecture, that the class of regular languages is closed under the Kleene star operation. That is: Let $M = (Q, \Sigma, \delta, q_0, F)$ be a DFA. Find an NFA $N$ which accepts $\mathcal{L}(M)^*$.
\sol % Put your solution here!

    
\item Find a regular expression for the language $\mathcal{L}_2 = \{ w: w$ has even length$\}$, over the alphabet $\Sigma = \{ 0, 1 \}$.
\sol % Put your solution here!

\item For the following regular expressions, give the state diagram of an NFA recognizing the same language. In all of the following, the alphabet is $\Sigma = \{ a, b \}$:  
\begin{enumerate}
	\item $(a \cup b)^* ((ab)b^* )$  
	\item $((a \cup b)(aa))^*$  
	\item $(a \cup b^*)a^*b^*$.  
\end{enumerate}

\sol % Put your solution here!
\begin{enumerate}
	\item Solution for (a)
	\item Solution for (b)
	\item Solution for (b)
\end{enumerate}

\item Show that $\mathcal{L}_4 = \{ a^i b^j : i \neq j \}$ is not a regular language.

\sol % Put your solution here!




%%% LaTeX HINTS:

% If you want to "draw" diagrams in LaTeX, you have two options:

%1. Use tikz. This can be hard, but you can look at the example from Pre-Work Lesson 2, copied over here.
%	\begin{figure}[ht]
%	\centering
%	\begin{tikzpicture}
%	\node[state, initial] (q0) {$q_0$};
%	\node[state, right of=q0] (q1) {$q_1$};
%	\node[state, accepting, right of=q1] (q2) {$q_2$};
%	\draw (q0) edge[above] node{0, 1} (q1)
%	(q1) edge[above] node{0, 1} (q2)
%	(q2) edge[bend left, below] node{0, 1} (q0);
%	\end{tikzpicture}
%	\end{figure}

%2. Draw it by hand. Take a good picture, crop it, and upload it to overleaf. Then use the following command:

% \includegraphics{filename-without-extension}
% Depending on the dimensions of the picture, you might need to rescale it:
% \includegraphics[scale=0.5]{filename-without-extension}.
% Sometimes I need to use scale=0.2 or 0.3 before the pictures display decently.

\end{enumerate}

\end{document}