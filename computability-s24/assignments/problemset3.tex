\documentclass[12pt]{article}
% 12pt font size, article class.
% Usually stick to this, though you can try "amsart" for an AMS style article instead.

\usepackage{amsmath,amssymb,psfrag,epsfig,boxedminipage,helvet,amsthm,endnotes,version,multicol}
\usepackage{pgfplots}
\usepackage{tikz}
\usetikzlibrary{automata, positioning, arrows}
% Package imports. We probably don't need all of them, but it doesn't hurt.

\title{Problem Set 3}
\author{Your Name Here!}
\date{}

\makeatletter
\let\newtitle\@title
\makeatother

\renewcommand{\thepage}{\footnotesize MAT 4520, Computability, \newtitle, p. \arabic{page}} 

\newcommand{\sol}{\par{\bf SOLUTION}: }

\begin{document}

\maketitle

\noindent
    {\bf Due: Monday, 3/18, at 11:59pm. }
    
\begin{enumerate}
\item Show that the class of context-free languages is closed under the Kleene star operation. 

\sol % Put your solution here!

\item A \textbf{term} in the first-order language of arithmetic is defined inductively as follows:
\begin{itemize}
	\item $0$ is a term,
	\item $1$ is a term,
	\item $x[w]$ is a term, whenever $w$ is a number (written in binary),
	\item if $t_1$ and $t_2$ are terms, then $t_1 + t_2$ is a term, and
	\item if $t_1$ and $t_2$ are terms, then $t_1 \times t_2$ is a term.
\end{itemize}

Show that $\{ t : t$ is a term $ \}$ is context-free.

\sol % Put your solution here!

\item A \textbf{formula} in the first-order language of arithmetic is defined inductively as follows:
\begin{itemize}
	\item if $t_1$ and $t_2$ are terms, then $t_1 = t_2$ is a formula,
	\item if $t_1$ and $t_2$ are terms, then $t_1 < t_2$ is a formula,
	\item if $\phi$ is a formula, then $\lnot \phi$ is a formula,
	\item if $\phi$ and $\psi$ are formulas, then $\phi \wedge \psi$ is a formula, and
	\item if $\phi$ is a formula, then $\exists x[w] \phi$ is a formula whenever $w$ is a number (written in binary).
\end{itemize}

Show that $\{ \phi : \phi$ is a formula $ \}$ is context-free.

\sol % Put your solution here!    

\item Give a formal state diagram of a Turing Machine which recognizes the language 
\begin{displaymath}
\{\# a^n + a^m = a^{n+m} : n , m \in \mathbb{N} \}.
\end{displaymath} (You can use the ``$\#$'' symbol to recognize the beginning of the tape.)

\sol % Put your solution here!

\item Sketch a proof that the class of computably enumerable languages is closed under intersection. (Hint: if $M_1$ and $M_2$ are two Turing machines, put \textbf{all} states of $M_1$ and all states of $M_2$ into your new machine, simulate $M_1$ on input $w$. What happens if $M_1$ loops forever? What happens if $M_1$ accepts? What happens if $M_1$ rejects?)

\sol % Put your solution.



%%% LaTeX HINTS:

% If you want to "draw" diagrams in LaTeX, you have two options:

%1. Use tikz. This can be hard, but you can look at the example from Pre-Work Lesson 2, copied over here.
%	\begin{figure}[ht]
%	\centering
%	\begin{tikzpicture}
%	\node[state, initial] (q0) {$q_0$};
%	\node[state, right of=q0] (q1) {$q_1$};
%	\node[state, accepting, right of=q1] (q2) {$q_2$};
%	\draw (q0) edge[above] node{0, 1} (q1)
%	(q1) edge[above] node{0, 1} (q2)
%	(q2) edge[bend left, below] node{0, 1} (q0);
%	\end{tikzpicture}
%	\end{figure}

%2. Draw it by hand. Take a good picture, crop it, and upload it to overleaf. Then use the following command:

% \includegraphics{filename-without-extension}
% Depending on the dimensions of the picture, you might need to rescale it:
% \includegraphics[scale=0.5]{filename-without-extension}.
% Sometimes I need to use scale=0.2 or 0.3 before the pictures display decently.

\end{enumerate}

\end{document}