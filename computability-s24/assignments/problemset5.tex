\documentclass[12pt]{article}
% 12pt font size, article class.
% Usually stick to this, though you can try "amsart" for an AMS style article instead.

\usepackage{amsmath,amssymb,psfrag,epsfig,boxedminipage,helvet,amsthm,endnotes,version,multicol}
\usepackage{pgfplots}
\usepackage{tikz}
\usetikzlibrary{automata, positioning, arrows}
% Package imports. We probably don't need all of them, but it doesn't hurt.

\title{Problem Set 5}
\author{Your Name Here!}
\date{}

\makeatletter
\let\newtitle\@title
\makeatother

\renewcommand{\thepage}{\footnotesize MAT 4520, Computability, \newtitle, p. \arabic{page}} 

\newcommand{\sol}{\par{\bf SOLUTION}: }

\begin{document}

\maketitle

\noindent
    {\bf Due: Thursday 5/2, at 11:59pm. }
    
\begin{enumerate}
	
\item Knowing that $SUBSET$-$SUM$ is NP-complete, show that $NONEMPTY$-$SUBSET$-$SUM$ is NP-complete, where $NONEMPTY$-$SUBSET$-$SUM$ is the following language:
\begin{displaymath}
\{ \langle x_1, \ldots, x_n, t \rangle : \exists \text{ a non-empty subset } S \subseteq \{ 1, \ldots n \} \text{ such that } \sum\limits_{i \in S} x_i = t \}.
\end{displaymath} You must show two things: first that this language is in NP, and second, that this langauge is NP-hard (ie, if we have a polynomial time algorithm which decides this language, then there is a polynomial time algorithm which decides any other language in NP). (Hint: show that if this problem is in $P$, then so is $SUBSET$-$SUM$)

\sol % Put your solution here

\item Suppose $SUBSET$-$SUM \in P$ and let $M$ be a polynomial-time algorithm which decides it. Find a polynomial-time algorithm which, given $\langle x_1, \ldots, x_n, t \rangle$, outputs a subset $S \subseteq \{ 1, \ldots, n \}$ such that $\sum\limits_{i \in S} x_i = t$, or outputs ``no solution'' if no such subset exists.

(That is: if we can decide if there is a subset of $\{ x_1, \ldots, x_n \}$ whose sum equals $t$ in polynomial time, show that we can actually find that subset whose sum equals $t$ in polynomial time also.)

\sol % Put your solution here

\item Answer both parts:
\begin{enumerate}
	\item Describe what it means for a language to be computable (decidable), computably enumerable (c.e.) but not computable, and co-c.e. but not computable.
	\item For each of the following languages, determine which, if any, of the above categories the language belongs to:
	\begin{enumerate}
		\item $\overline{H_{\text{TM}}}$, where $H_\text{TM} = \{ \langle M, w \rangle : M $ halts on input $ w \}$
		\item $CLIQUE$
		\item $A_\text{TM}$ (the acceptance problem)
	\end{enumerate}
\end{enumerate}

\sol
\begin{enumerate}
	\item Solution for (a)
	\item Solutions for (b) below:
	\begin{enumerate}
		\item $\overline{H_{\text{TM}}}$
		\item $CLIQUE$
		\item $A_{\text{TM}}$
	\end{enumerate}
\end{enumerate}

%%% LaTeX HINTS:

% If you want to "draw" diagrams in LaTeX, you have two options:

%1. Use tikz. This can be hard, but you can look at the example from Pre-Work Lesson 2, copied over here.
%	\begin{figure}[ht]
%	\centering
%	\begin{tikzpicture}
%	\node[state, initial] (q0) {$q_0$};
%	\node[state, right of=q0] (q1) {$q_1$};
%	\node[state, accepting, right of=q1] (q2) {$q_2$};
%	\draw (q0) edge[above] node{0, 1} (q1)
%	(q1) edge[above] node{0, 1} (q2)
%	(q2) edge[bend left, below] node{0, 1} (q0);
%	\end{tikzpicture}
%	\end{figure}

%2. Draw it by hand. Take a good picture, crop it, and upload it to overleaf. Then use the following command:

% \includegraphics{filename-without-extension}
% Depending on the dimensions of the picture, you might need to rescale it:
% \includegraphics[scale=0.5]{filename-without-extension}.
% Sometimes I need to use scale=0.2 or 0.3 before the pictures display decently.

\end{enumerate}

\end{document}